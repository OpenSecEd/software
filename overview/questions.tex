\question[3]\label{q:software}
% examgen: software:A
Can a files such as images (e.g.\ JPEGs) and other data be dangerous?

\begin{solution}
  Yes, they can contain machine code which can be executed if there is e.g.\ 
  a buffer overrun vulnerability in the software that reads the data.
\end{solution}


\question[3]\label{q:software}
% examgen: software:E:C
We have talked about how the users' mental models of how a program (and 
computer) works can endanger the users' security when the mental model and 
reality are not aligned.
This is true also for developers (we mentioned this when we talked about 
software security), give an example of how the developers' mental models are 
relevant for software security.

\begin{solution}
  Gollmann talked about broken abstractions.
  One example is characters: usually we abstract away the encoding and decoding 
  parts, we see them as characters and not bytes.
  So encodings like UTF-8 can cause problems since the same character can be 
  represented by several byte sequences.

  Another is the finite precision arithmetic that we work with in computers, 
  usually congruences modulo \(2^{32}\) or \(2^{64}\).
\end{solution}


\question[3]\label{q:software}
% examgen: software:E:C:A
Give an example where \enquote{data} can be mistaken for \enquote{code}.

\begin{solution}
  Shell scripting is an easy example.
  Here you can store part of the code in variables, the simply substitute them.
  Consider the following \texttt{/bin/echo -e \$\{1\}}.
  The variable \texttt{\$\{1\}} will be substituted and the result will be 
  interpreted as code.
\end{solution}


\question\label{q:software}
% examgen: software:E:C
Describe the three malware reproduction techniques
\begin{parts}
  \part[1] virus,
  \begin{solution}
    The virus inserts its own code into other programs code.
    When the other programs are run the virus' payload is run too and the 
    infection can spread further.
  \end{solution}

  \part[1] worm,
  \begin{solution}
    The worm spreads by its own means, e.g.\ by utilizing networks (shared file 
    systems, remote executions bugs in network services) or emailing itself 
    using available programs on the computer.
  \end{solution}

  \part[1] trojan horse.
  \begin{solution}
    The trojan horse is a legitimate-looking program which contains unwanted 
    functionality.
    E.g.\ it is a flash-light app, but in the background it uploads the contact 
    list to the app's developer.
  \end{solution}
\end{parts}


\question\label{q:software}
% examgen: software:E:C
Describe the three malware payload types
\begin{parts}
  \part[1] root-kit,
  \begin{solution}
    The root-kit is designed to fully compromise the system, gaining root 
    privileges.
    Either it provides a back-door or it simply make predefined modifications 
    to the system behaviour.
    It usually modifies the operating system to stay hidden from the user.
  \end{solution}

  \part[1] spyware,
  \begin{solution}
    Spyware records information about the user of the system, e.g.\ logging 
    key-presses or getting the browser history.
  \end{solution}

  \part[1] ransomware.
  \begin{solution}
    Ransomware encrypts the files in the file system.
    Then it asks the user for a ransom: the user has to pay to get the 
    decryption key.
  \end{solution}
\end{parts}
