% $Id$
%\documentclass[handout]{beamer}
\documentclass{beamer}
\usepackage[utf8]{inputenc}
\usepackage[T1]{fontenc}
\usepackage[swedish,english]{babel}
\usepackage{url}
\usepackage{color}
\usepackage{multicol}
\usepackage{listings}
\usepackage[natbib,style=alphabetic,maxbibnames=99]{biblatex}
\addbibresource{overview.bib}
\setbeamertemplate{bibliography item}[text]

\lstset{%
  basicstyle=\small,
  numberstyle=\small,
  numbers=left,
  stepnumber=1,
  numbersep=5pt,
  showspaces=false,
  showstringspaces=true,
  showtabs=false,
  frame=TB,
  tabsize=2,
  captionpos=b,
  breaklines=true,
  breakatwhitespace=true,
}
\lstdefinelanguage{none}{%
  keywords={}
}
\definecolor{termbkg}{gray}{0.90}%
\definecolor{textbkg}{gray}{0.97}%
\definecolor{commentcol}{rgb}{0,0.5,0}%
\lstdefinestyle{term}{%
  language=none,
  numbers=left,
  frame=single,
  basicstyle=\ttfamily\small,
  backgroundcolor=\color{termbkg},
}
\lstdefinestyle{conf}{%
  language=none,
  numbers=none,
  frame=single,
  basicstyle=\ttfamily\small,
  backgroundcolor=\color{textbkg},
}
\lstdefinestyle{text}{%
  language=none,
  numbers=none,
  frame=single,
  basicstyle=\ttfamily\small,
  backgroundcolor=\color{textbkg},
}
\lstdefinestyle{code}{%
  numbers=left,
  frame=single,
  basicstyle=\ttfamily\small,
  backgroundcolor=\color{white},
}
\lstset{style=text}
\lstnewenvironment{src}[1][]{%
  \lstset{style=code,#1}
}{%
  \lstset{style=text}
}
\lstnewenvironment{config}[1][]{%
  \lstset{style=conf,#1}
}{%
  \lstset{style=conf}
}
\lstnewenvironment{terminal}[1][]{%
  \lstset{style=term,#1}
}{%
  \lstset{style=text}
}
\newcommand{\code}[2][style=code]{\lstinline[#1]`#2`}
\newcommand{\term}[2][style=term]{\lstinline[#1]`#2`}

\mode<presentation>{%
  \usetheme{Frankfurt}
  \setbeamercovered{transparent}
  \usecolortheme{seagull}
}
\setbeamertemplate{footline}{\insertframenumber}

\title{%
  Software Security
}
\author{Daniel Bosk\footnote{%
  This work is licensed under the Creative Commons Attribution-ShareAlike 3.0 
  Unported license.
	To view a copy of this license, visit 
	\url{http://creativecommons.org/licenses/by-sa/3.0/}.
}}
\institute[MIUN IKS]{%
  Department of Information and Communication Systems,\\
  Mid Sweden University, SE-851\,70 Sundsvall
}
\date{\today}

%\pgfdeclareimage[height=0.65cm]{university-logo}{MU_logotyp_int_CMYK.pdf}
%\logo{\pgfuseimage{university-logo}}

\AtBeginSection[]{%
  \begin{frame}<beamer>{Overview}
    \begin{multicols}{2}
      \tableofcontents[currentsection]
    \end{multicols}
  \end{frame}
}
%\AtBeginSubsection[]{%
%  \begin{frame}<beamer>{Overview}
%    \tiny
%    \tableofcontents[currentsubsection,sectionstyle=shaded]
%  \end{frame}
%}

\begin{document}

\begin{frame}
  \titlepage
\end{frame}

\begin{frame}{Overview}
  \begin{multicols}{2}
    \tableofcontents
  \end{multicols}
\end{frame}


% Since this a solution template for a generic talk, very little can
% be said about how it should be structured. However, the talk length
% of between 15min and 45min and the theme suggest that you stick to
% the following rules:  

% - Exactly two or three sections (other than the summary).
% - At *most* three subsections per section.
% - Talk about 30s to 2min per frame. So there should be between about
%   15 and 30 frames, all told.


\section{Introduction}

\subsection{Security and Reliability}

\begin{frame}{\insertsubsectionhead}
  \begin{itemize}
    \item As long as our computer is offline, used only by ourselves, and we 
      don't add any accessories (e.g.\ USB devices~\cite{ieeespectrum2014usb}), 
      then we don't have any problems.

    \item Problems start to occur when other users start using our software (in 
      some way), then input to our programs isn't necessarily what we expect.

  \end{itemize}
\end{frame}

\begin{frame}{\insertsubsectionhead}
  \begin{itemize}
    \item Software reliability concerns software quality in the sense of 
      accidental failures, i.e.\ the assumption that input is benign.

    \item Software security concerns software quality in the sense of 
      intentional failures, i.e.\ the assumption that input is malign.

    \item We will focus on the latter.

  \end{itemize}
\end{frame}

% XXX add better red line to malware
\subsection{Malware}

\begin{frame}{\insertsubsectionhead}
  \begin{definition}[Malware]
    Comes from \emph{malicious software} and means software with a malicious 
    intent.
  \end{definition}
  \begin{definition}[Computer Virus]
    A form of malware which has self-replicating code.
    It \emph{infects} other programs by inserting itself into their program 
    code, and in turn when these programs are run the virus payload is run to 
    replicate even further.
  \end{definition}
  \begin{definition}[Worm]
    A form of malware which replicates itself, not by infection, but by copying 
    itself to different disks, via networks, or even emailing itself 
    automatically to everyone in the user's contact list.
  \end{definition}
\end{frame}

\begin{frame}{\insertsubsectionhead}
  \begin{definition}[Trojan Horse]
    A form of malware which acts as a legitimate program but has hidden 
    features which are malicious, e.g.\ a utility program which steals your 
    login credentials in the background.
  \end{definition}
  \begin{definition}[Logic Bomb]
    A form of malware which resides doing nothing until a logical condition is 
    satisfied, then it executes its malicious code -- e.g.\ erasing all files 
    etc.
  \end{definition}
\end{frame}

% XXX add better red line to changes
\subsection{Change in Environment}

\begin{frame}{\insertsubsectionhead}
  \begin{itemize}
    \item Change is one of the dangers to security.

    \item There are systems which are designed to be secure, and actually are 
      secure, but then \dots

    \item upgrades are needed, or not needed but wanted.

    \item This might come in the form of updating a component or utilising the 
      system in an environment it wasn't designed for.

  \end{itemize}
\end{frame}


\section{Broken Abstractions}

\subsection{Numbers and Characters}

\begin{frame}{\insertsubsectionhead}
  \begin{itemize}
    \item Imagine we want to keep the user in the directory \code{''/A/B/C''}.

    \item Our program implements this by taking the name of the input file as 
      input from the user.

    \item Then to access the file it opens \code{''/A/B/C/''+filename}.

    \item What if the user inputs \code{filename = ''../../../etc/passwd''}?

    \item Then this would evaluate to opening 
      \code{/A/B/C/../../../etc/passwd}.
  \end{itemize}
\end{frame}

\begin{frame}{\insertsubsectionhead}
  \begin{itemize}
    \item Fine, we ban the string \code{''../''}.

    \item Then what about \code{''..\%c0\%af..''}?

  \end{itemize}
\end{frame}

\begin{frame}{\insertsubsectionhead}
  \begin{itemize}
    \item All character representations in the computer comes in the form of 
      different encodings, e.g.\ UTF-8 encoding.

    \item The decoders might be programmed differently, some takes into account 
      the errors in different encoders to compensate -- and this can be 
      exploited.

    \item Where the encoding is done can also be exploited.

  \end{itemize}
\end{frame}

% XXX add more details on UTF-8 coding

% XXX add more examples on integer overflows
\begin{frame}[fragile]{\insertsubsectionhead}
  \begin{itemize}
    \item Integer overflows is another problem.

    \item Consider the following example.

  \end{itemize}
  \begin{src}[language=C]
char buf[128];

void
combine( char *s1, size_t len1, char *s2, size_t len2)
{
  if ( len1 + len2 + 1 <= sizeof(buf) ) {
    strncpy( buf, s1, len1 );
    strncat( buf, s2, len2 );
  }
}
  \end{src}
\end{frame}

\begin{frame}{\insertsubsectionhead}
  \begin{itemize}
    \item Let \code{len2} be very long, say \(2^{32} - 1\), i.e.\ \code{len2 
      = 0xffffffff}.

    \item Now we have \(\text{\code{len1}} + \text{\code{len2}} 
      + 1 \pmod{2^{32}} = \text{\code{len1}} + 2^{32} - 1 + 1 \pmod{2^{32}} 
      = \text{\code{len1}} \pmod{2^{32}} < \) \code{sizeof(buf)}.
  \end{itemize}
\end{frame}

% XXX add more details and other examples of composition
\subsection{Function Composition}

\begin{frame}{\insertsubsectionhead}
  \begin{itemize}
    \item The login(1) and rlogin(1) composition bug was found in Linux and AIX 
      systems which didn't check the syntax of the username.

    \item The syntax of login(1) is \code{login [-p] [-h host] [[-f] user]}.

    \item The syntax of rlogin(1) is \code{rlogin [-l user] machine}.

    \item rlogin(1) connects to the machine and runs \code{login user machine}.

    \item However, the user could be chosen to be ''-froot''.
  \end{itemize}
\end{frame}

% XXX add canonical representations
%\subsection{Canonical Representations}
%
%\begin{frame}{\insertsubsectionhead}
%\end{frame}


\section{Memory Management}

\subsection{Memory Structure}

% XXX clarify memory structure with figure
\begin{frame}{\insertsubsectionhead}
  \begin{itemize}
    \item We have the code of the program.
    \item We have some program data.
    \item We have a stack growing downwards.
    \item We have a heap growing upwards.
  \end{itemize}
\end{frame}

\subsection{Overruns}

% XXX add more description of buffer overruns
\begin{frame}{\insertsubsectionhead}
  \begin{itemize}
    \item Buffer overruns
    \item Stack overruns
    \item Heap overruns
    \item All variables in a program use storage from either the stack or heap.
  \end{itemize}
\end{frame}

\begin{frame}[fragile]{\insertsubsectionhead}
  \begin{src}[language=C]
int
login( void )
{
  char correct_password[] = "swordfish";
  char user_password[16] = {0};

  printf( "user password: ");
  fscanf( "\%s", user_password );

  if ( !strcmp( correct_password, user_password ) )
    return 0;
  return 1;
}
  \end{src}
\end{frame}

% XXX add figure for previous example

% XXX add more examples of overruns

%\subsection{Double-Free Vulnerabilities}
%
%\begin{frame}{\insertsubsectionhead}
%\end{frame}

\subsection{Type Confusion}

% XXX clarify slide on type confusion
\begin{frame}{\insertsubsectionhead}
  \begin{itemize}
    \item There are some problems in object-oriented languages too.
    \item Trick the system to point to a different memory location.
    \item Thus a write using one type actually modifies something believed to 
      be of another type somewhere else.
  \end{itemize}
\end{frame}


\section{Data and Code}

\subsection{Scripting}

% XXX add better description of scripting vuln

\begin{frame}[fragile]{\insertsubsectionhead}
  \begin{src}
    cat thefile | mail addresses
  \end{src}
  \begin{itemize}
    \item What happens with the address \code{foo@bar.org | rm -Rf /}?
  \end{itemize}
\end{frame}

\subsection{SQL Injection}

% XXX add better description and examples of SQL injection

\begin{frame}[fragile]{\insertsubsectionhead}
  \begin{src}
    \$sql = "SELECT * FROM client WHERE name = '\$name'"
  \end{src}
  \begin{itemize}
    \item Insert the name \code{Eve' OR 1=1--}.
    \item This will get a totally different meaning.
  \end{itemize}
\end{frame}


% XXX add defences to software security, perhaps inline
% XXX instead of separate section

%\section{Defences}
%
%\subsection{Prevention}
%
%\subsection{Hardware}
%
%\begin{frame}{\insertsubsectionhead}
%  \begin{itemize}
%    \item Hardware
%    %\item Modus Operandi
%    \item Safer Functions
%    \item Filtering
%    \item Type Safety
%  \end{itemize}
%\end{frame}
%
%%\begin{frame}{\insertsubsectionhead}
%%\end{frame}
%%
%%\begin{frame}{\insertsubsectionhead}
%%\end{frame}
%%
%%\begin{frame}{\insertsubsectionhead}
%%\end{frame}
%%
%%\begin{frame}{\insertsubsectionhead}
%%\end{frame}
%
%\subsection{Detection}
%
%\begin{frame}{\insertsubsectionhead}
%  \begin{itemize}
%    \item Canaries
%    \item Code Inspection
%    \item Testing
%  \end{itemize}
%\end{frame}
%
%%\begin{frame}{\insertsubsectionhead}
%%\end{frame}
%%
%%\begin{frame}{\insertsubsectionhead}
%%\end{frame}
%
%\subsection{Mitigation}
%
%\begin{frame}{\insertsubsectionhead}
%  \begin{itemize}
%    \item Least privilege \dots
%  \end{itemize}
%\end{frame}
%
%\subsection{Reaction}
%
%\begin{frame}{\insertsubsectionhead}
%  \begin{itemize}
%    \item Keep up to date \dots
%  \end{itemize}
%\end{frame}

%%%%%%%%%%%%%%%%%%%%%%

\begin{frame}{Referenser}
  \small
  \printbibliography
\end{frame}

\end{document}
