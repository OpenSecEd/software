\mode*

% Since this a solution template for a generic talk, very little can
% be said about how it should be structured. However, the talk length
% of between 15min and 45min and the theme suggest that you stick to
% the following rules:  

% - Exactly two or three sections (other than the summary).
% - At *most* three subsections per section.
% - Talk about 30s to 2min per frame. So there should be between about
%   15 and 30 frames, all told.


\section{Memory Management}

\subsection{Memory Structure}

\begin{frame}
  \includegraphics[height=\textheight]{procmem.jpg}
\end{frame}

\subsection{Overruns}

% XXX add more description of buffer overruns
\begin{frame}
  \begin{itemize}
    \item Buffer overruns
      \begin{itemize}
        \item Stack overruns
        \item Heap overruns
      \end{itemize}

    \item All variables in a program use storage from either the stack or heap.
  \end{itemize}
\end{frame}

\begin{frame}[fragile]
  \inputminted{C}{login.c}
\end{frame}

% XXX add figure for previous example

% XXX add more examples of overruns

%\subsection{Double-Free Vulnerabilities}
%
%\begin{frame}
%\end{frame}

\subsection{Type Confusion}

% XXX clarify slide on type confusion
\begin{frame}
  \begin{itemize}
    \item There are some problems in object-oriented languages too.
    \item Trick the system to point to a different memory location.
    \item Thus a write using one type actually modifies something believed to 
      be of another type somewhere else.
  \end{itemize}
\end{frame}


%\section{Malware}
%
%% XXX add better storyline to malware
%\subsection{Background}
%
%\begin{frame}
%  \begin{itemize}
%    \item Comes from \emph{malicious software} and means software with 
%      a malicious intent.
%
%    \item In the early days they were mostly experiments or pranks.
%
%    \item Today they are mostly used for special purposes:
%      \begin{itemize}
%        \item steal personal, financial or business information,
%        \item cripple competition,
%        \item etc.
%      \end{itemize}
%
%  \end{itemize}
%\end{frame}
%
%\begin{frame}
%  \begin{itemize}
%    \item There are many types of malware.
%
%    \item Their classification depends on the largest threat vector.
%
%  \end{itemize}
%\end{frame}
%
%\subsection{Malware Types}
%
%\begin{frame}[allowframebreaks]
%  \begin{description}
%    \item[Computer Virus]
%      A form of malware which has self-replicating code.
%      It \emph{infects} other programs by inserting itself into their program 
%      code, and in turn when these programs are run the virus payload is run to 
%      replicate even further.
%
%    \item[Worm]
%      A form of malware which replicates itself, not by infection, but by 
%      copying itself to different disks, via networks, or even emailing itself 
%      automatically to everyone in the user's contact list.
%
%    \item[Trojan Horse]
%      A form of malware which acts as a legitimate program but has hidden 
%      features which are malicious, e.g.\ a utility program which steals your 
%      login credentials in the background or simply acts as a backdoor.
%      Usually used in combination of social engineering.
%
%    \item[Rootkit]
%      A piece of software designed to provide access that would otherwise be 
%      restricted.
%      It also keeps well-hidden and is notoriously difficult to detect and 
%      remove.
%      Usually this comes from modifying the operating system.
%
%    \item[Spyware]
%      This software simply tries to gather information about a target without 
%      their knowledge.
%      Usually the collected information is sent to a third party.
%
%      Keylogging falls under this category.
%
%    \item[Adware]
%      This is simply a type of malware that presents advertisements to the user
%      of the infected system.
%      Obviously staying undetected is not an option, so making itself difficult
%      to remove is the strategy of choice.
%
%    \item[Scareware]
%      This is a type of malware that uses social engineering to trick users to 
%      buy unwanted software, e.g.\ fake antivirus software.
%
%    \item[Ransomware]
%      This is a type of malware that restricts the users access to the system.
%      A common technique is to encrypt all the user's files.
%      Then the user is presented with the option of buying the decryption key 
%      for bitcoins.
%
%      They typically propagate as trojans.
%
%  \end{description}
%\end{frame}



% XXX add defences to software security, perhaps inline
% XXX instead of separate section

%\section{Defences}
%
%\subsection{Prevention}
%
%\subsection{Hardware}
%
%\begin{frame}
%  \begin{itemize}
%    \item Hardware
%    %\item Modus Operandi
%    \item Safer Functions
%    \item Filtering
%    \item Type Safety
%  \end{itemize}
%\end{frame}
%
%%\begin{frame}
%%\end{frame}
%%
%%\begin{frame}
%%\end{frame}
%%
%%\begin{frame}
%%\end{frame}
%%
%%\begin{frame}
%%\end{frame}
%
%\subsection{Detection}
%
%\begin{frame}
%  \begin{itemize}
%    \item Canaries
%    \item Code Inspection
%    \item Testing
%  \end{itemize}
%\end{frame}
%
%%\begin{frame}
%%\end{frame}
%%
%%\begin{frame}
%%\end{frame}
%
%\subsection{Mitigation}
%
%\begin{frame}
%  \begin{itemize}
%    \item Least privilege \dots
%  \end{itemize}
%\end{frame}
%
%\subsection{Reaction}
%
%\begin{frame}
%  \begin{itemize}
%    \item Keep up to date \dots
%  \end{itemize}
%\end{frame}

