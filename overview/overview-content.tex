\title{%
  Software Security
}
\author{%
  Daniel Bosk
}
\institute[MIUN IKS]{%
  Department of Information and Communication Systems,\\
  Mid Sweden University, SE-851\,70 Sundsvall
}
\date{\today}

\mode<presentation>{%
  \begin{frame}
    \titlepage
  \end{frame}
}
\mode<article>{%
  \maketitle
}

\mode*

\begin{abstract}
  Perhaps the part of security most people intuitively associate with security, 
and computer security in particular, is software security.
This part of computer security treats vulnerabilities in software, e.g.\ buffer 
overruns or code injections.
This is a very important part of security, because although the design is 
flawless, its implementation might have vulnerabilities.
As an example, most phones are designed to keep the user and applications 
unpriviledged, thus all applications will run with the principle of least 
priviledges and compartmentalized from each other.
However, software bugs in the operating system can allow malicious apps to gain
priviledges to e.g.\ monitor other apps.

After this session you should be able to
\begin{itemize}
  \item \emph{understand} the the need to consider software security in 
    software development.
  \item \emph{evaluate} the software security requirements for different 
    sitations.
\end{itemize}

Gollmann treats this area in Chapter 10 of his book, 
\citetitle{Gollmann2011cs}~\cite{Gollmann2011cs}.
The recommended exercises to do after reading this material are 10.1, 10.3 and 
10.4 in~\cite{Gollmann2011cs}.
Anderson also treats this subject --- in Chapter 4.4 and Chapter 18 of 
\citetitle{Anderson2008sea}~\cite{Anderson2008sea} --- albeit with less 
technical details.
We also treat the results of \citetitle{BSIMMFindings}~\cite{BSIMMFindings}.

\end{abstract}

% Since this a solution template for a generic talk, very little can
% be said about how it should be structured. However, the talk length
% of between 15min and 45min and the theme suggest that you stick to
% the following rules:  

% - Exactly two or three sections (other than the summary).
% - At *most* three subsections per section.
% - Talk about 30s to 2min per frame. So there should be between about
%   15 and 30 frames, all told.


\section{Introduction}

\subsection{Security and Reliability}

\begin{frame}
  \begin{itemize}
    \item As long as our computer is offline, used only by ourselves, and we 
      don't add any accessories (e.g.\ USB devices~\cite{ieeespectrum2014usb}), 
      then we don't have any problems.

      \pause{}

    \item Problems start to occur when other users start using our software (in 
      some way), then input to our programs isn't necessarily what we expect.

  \end{itemize}
\end{frame}

\begin{frame}
  \begin{description}
    \item[Software reliability] This concerns software quality in the sense of 
      accidental failures, i.e.\ the assumption that input is benign.

      \pause{}

    \item[Software security] This concerns software quality in the sense of 
      intentional failures, i.e.\ the assumption that input is malign.

  \end{description}
\end{frame}

\subsection{Changes}

% XXX add better storyline to changes

\begin{frame}
  \begin{itemize}
    \item Change is one of the dangers to security.

    \item There are systems which are designed to be secure, and actually are 
      secure, but then \dots

    \item upgrades are needed, or not needed but wanted.

    \item This might come in the form of updating a component or utilizing the 
      system in an environment it wasn't designed for.

  \end{itemize}
\end{frame}


\section{Broken Abstractions}

\subsection{File System Paths}

\begin{frame}[fragile]
  \inputminted{python}{jail.py}
\end{frame}

\begin{frame}[fragile]
  \begin{example}[./jail.py ../../etc/passwd]
    \begin{pycode}
import jail
jail.main(["jailopen", "../../etc/passwd"])
    \end{pycode}
  \end{example}
\end{frame}

\begin{frame}
  \pyc[variable]{import os}
  \begin{alertblock}{The Problem: Abstraction of paths}
    \begin{itemize}
      \item We had \pyb[variable]{JAIL_PATH = os.environ["HOME"]}.
      \item We let \pyb[variable]{filename = "../../etc/passwd"}.
      \item Thus the file we open is \pyb[variable]{JAIL_PATH + "/" + filename} 
        which results in \pyc[variable]{print(JAIL_PATH + "/" + filename)}.
      \item Hence we actually read /etc/passwd.
    \end{itemize}
  \end{alertblock}
\end{frame}

\begin{frame}
  \begin{itemize}
    \item Fine, we ban the string \mintinline{python}{"../"}.

    \item Then what about \mintinline{python}{"..\%c0\%af.."}?

  \end{itemize}
\end{frame}

\subsection{Character Encoding}

\begin{frame}
  \begin{itemize}
    \item All character representations in the computer comes in the form of 
      different encodings, e.g.\ UTF-8 encoding.

    \item The decoders might be programmed differently, some takes into account 
      the errors in different encoders to compensate -- and this can be 
      exploited.

    \item Where the encoding and decoding is done can also be exploited.

  \end{itemize}
\end{frame}

\begin{frame}
  \begin{block}{UTF-8}
  \end{block}
\end{frame}

% XXX add more details on UTF-8 coding

\subsection{Integer Overflows}

% XXX add more examples on integer overflows
\begin{frame}[fragile]
  \inputminted{C}{combine.c}
\end{frame}

\begin{frame}
  \begin{alertblock}{The Problem: Abstraction of integers}
    \begin{itemize}
      \item Let \mintinline{C}{len2} be very long, say \(2^{32} - 1\), i.e.\ 
        \mintinline{C}{len2 = 0xffffffff}.

      \item Now we have
        \begin{align*}
          \text{\mintinline{C}{len1}} + \text{\mintinline{C}{len2}} 
          + 1 \pmod{2^{32}}
          &= \text{\mintinline{C}{len1}} + 2^{32} - 1 + 1 \pmod{2^{32}} \\
          &= \text{\mintinline{C}{len1}} \pmod{2^{32}} \\
          &< \text{\mintinline{C}{sizeof(buf)}}.
        \end{align*}

      \item Thus we pass the test, although we shouldn't.
    \end{itemize}
  \end{alertblock}
\end{frame}

\begin{frame}
  \begin{remark}
    This is worse if we use \emph{signed} integers \dots
  \end{remark}
\end{frame}

% XXX add more details and other examples of composition
\subsection{Data and Code}

\begin{frame}[fragile]
  \begin{example}[echo.sh "-E test\textbackslash ning"]
    \inputminted{sh}{echo.sh}
    \begin{pycode}[echo.sh]
import subprocess
proc = subprocess.Popen(["./echo.sh", "-E test\\ning"], \
  stdout=subprocess.PIPE)
print("\\begin{verbatim}" + proc.stdout.read().decode("utf-8") + \
  "\\end{verbatim}")
    \end{pycode}
  \end{example}
\end{frame}

\begin{frame}[fragile]
  \begin{example}[echofix.sh "-E test\textbackslash ning"]
    \inputminted{sh}{echofix.sh}
    \begin{pycode}[echofix.sh]
import subprocess
proc = subprocess.Popen(["./echofix.sh", "-E test\\ning"], \
  stdout=subprocess.PIPE)
print("\\begin{verbatim}" + proc.stdout.read().decode("utf-8") + \
  "\\end{verbatim}")
    \end{pycode}
  \end{example}
\end{frame}

\begin{frame}
  \begin{itemize}
    \item The login(1) and rlogin(1) composition bug was found in Linux and AIX 
      systems which didn't check the syntax of the username.

    \item The syntax of login(1) is \mintinline{sh}{login [-p] [-h host] [[-f] 
        user]}.

    \item The syntax of rlogin(1) is \mintinline{sh}{rlogin [-l user] machine}.

    \item rlogin(1) connects to the machine and runs \mintinline{sh}{login user 
        machine}.

    \item However, the user could be chosen to be \enquote{-froot}.
  \end{itemize}
\end{frame}

% XXX add canonical representations
%\subsection{Canonical Representations}
%
%\begin{frame}
%\end{frame}

% XXX add better description of scripting vuln

\begin{frame}[fragile]
  \begin{minted}{sh}
    cat ${1} | mail ${2}
  \end{minted}
  \begin{itemize}
    \item What happens with the address
      \mintinline{sh}{"foo@bar.org | rm -Rf /"}?
  \end{itemize}
\end{frame}

% XXX add better description and examples of SQL injection

\begin{frame}[fragile]
  \begin{minted}[startinline]{php}
    $sql = "SELECT * FROM client WHERE name = '$name'"
  \end{minted}
  \begin{itemize}
    \item Insert the name \mintinline[startinline]{php}{Eve' OR 1=1--}.
    \item This will get a totally different meaning.
  \end{itemize}
\end{frame}

\begin{frame}
  \begin{figure}
    \centering
    \includegraphics[width=\textwidth]{BobbyTables.png}
    \caption{%
      XKCD's Exploits of a Mom.
      Image: \cite{BobbyTables}.
    }
  \end{figure}
\end{frame}


\section{Memory Management}

\subsection{Memory Structure}

\begin{frame}
  \includegraphics[height=\textheight]{procmem.jpg}
\end{frame}

\subsection{Overruns}

% XXX add more description of buffer overruns
\begin{frame}
  \begin{itemize}
    \item Buffer overruns
      \begin{itemize}
        \item Stack overruns
        \item Heap overruns
      \end{itemize}

    \item All variables in a program use storage from either the stack or heap.
  \end{itemize}
\end{frame}

\begin{frame}[fragile]
  \inputminted{C}{login.c}
\end{frame}

% XXX add figure for previous example

% XXX add more examples of overruns

%\subsection{Double-Free Vulnerabilities}
%
%\begin{frame}
%\end{frame}

\subsection{Type Confusion}

% XXX clarify slide on type confusion
\begin{frame}
  \begin{itemize}
    \item There are some problems in object-oriented languages too.
    \item Trick the system to point to a different memory location.
    \item Thus a write using one type actually modifies something believed to 
      be of another type somewhere else.
  \end{itemize}
\end{frame}


\section{Malware}

% XXX add better storyline to malware
\subsection{Background}

\begin{frame}
  \begin{itemize}
    \item Comes from \emph{malicious software} and means software with 
      a malicious intent.

    \item In the early days they were mostly experiments or pranks.

    \item Today they are mostly used for special purposes:
      \begin{itemize}
        \item steal personal, financial or business information,
        \item cripple competition,
        \item etc.
      \end{itemize}

  \end{itemize}
\end{frame}

\begin{frame}
  \begin{itemize}
    \item There are many types of malware.

    \item Their classification depends on the largest threat vector.

  \end{itemize}
\end{frame}

\subsection{Malware Types}

\begin{frame}[allowframebreaks]
  \begin{description}
    \item[Computer Virus]
      A form of malware which has self-replicating code.
      It \emph{infects} other programs by inserting itself into their program 
      code, and in turn when these programs are run the virus payload is run to 
      replicate even further.

    \item[Worm]
      A form of malware which replicates itself, not by infection, but by 
      copying itself to different disks, via networks, or even emailing itself 
      automatically to everyone in the user's contact list.

    \item[Trojan Horse]
      A form of malware which acts as a legitimate program but has hidden 
      features which are malicious, e.g.\ a utility program which steals your 
      login credentials in the background or simply acts as a backdoor.
      Usually used in combination of social engineering.

    \item[Rootkit]
      A piece of software designed to provide access that would otherwise be 
      restricted.
      It also keeps well-hidden and is notoriously difficult to detect and 
      remove.
      Usually this comes from modifying the operating system.

    \item[Spyware]
      This software simply tries to gather information about a target without 
      their knowledge.
      Usually the collected information is sent to a third party.

      Keylogging falls under this category.

    \item[Adware]
      This is simply a type of malware that presents advertisements to the user
      of the infected system.
      Obviously staying undetected is not an option, so making itself difficult
      to remove is the strategy of choice.

    \item[Scareware]
      This is a type of malware that uses social engineering to trick users to 
      buy unwanted software, e.g.\ fake antivirus software.

    \item[Ransomware]
      This is a type of malware that restricts the users access to the system.
      A common technique is to encrypt all the user's files.
      Then the user is presented with the option of buying the decryption key 
      for bitcoins.

      They typically propagate as trojans.

  \end{description}
\end{frame}



% XXX add defences to software security, perhaps inline
% XXX instead of separate section

%\section{Defences}
%
%\subsection{Prevention}
%
%\subsection{Hardware}
%
%\begin{frame}
%  \begin{itemize}
%    \item Hardware
%    %\item Modus Operandi
%    \item Safer Functions
%    \item Filtering
%    \item Type Safety
%  \end{itemize}
%\end{frame}
%
%%\begin{frame}
%%\end{frame}
%%
%%\begin{frame}
%%\end{frame}
%%
%%\begin{frame}
%%\end{frame}
%%
%%\begin{frame}
%%\end{frame}
%
%\subsection{Detection}
%
%\begin{frame}
%  \begin{itemize}
%    \item Canaries
%    \item Code Inspection
%    \item Testing
%  \end{itemize}
%\end{frame}
%
%%\begin{frame}
%%\end{frame}
%%
%%\begin{frame}
%%\end{frame}
%
%\subsection{Mitigation}
%
%\begin{frame}
%  \begin{itemize}
%    \item Least privilege \dots
%  \end{itemize}
%\end{frame}
%
%\subsection{Reaction}
%
%\begin{frame}
%  \begin{itemize}
%    \item Keep up to date \dots
%  \end{itemize}
%\end{frame}

%%%%%%%%%%%%%%%%%%%%%%

\begin{frame}
  \small
  \printbibliography{}
\end{frame}
