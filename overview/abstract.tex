Perhaps the part of security most people intuitively associate with security, 
and computer security in particular, is software security.
This part of computer security treats vulnerabilities in software, e.g.\ buffer 
overruns or code injections.
This is a very important part of security, because although the design is 
flawless, its implementation might have vulnerabilities.
As an example, most phones are designed to keep the user and applications 
unpriviledged, thus all applications will run with the principle of least 
priviledges and compartmentalized from each other.
However, software bugs in the operating system can allow malicious apps to gain
priviledges to e.g.\ monitor other apps.

After this session you should be able to
\begin{itemize}
  \item \emph{understand} the need to consider software security in software 
    development.
  \item \emph{evaluate} the software security requirements for different 
    sitations.
\end{itemize}

Gollmann treats this area in Chapter 10 of his book, 
\citetitle{Gollmann2011cs}~\cite{Gollmann2011cs}.
The recommended exercises to do after reading this material are 10.1, 10.3 and 
10.4 in~\cite{Gollmann2011cs}.
Anderson also treats this subject --- in Chapter 4.4 and Chapter 18 of 
\citetitle{Anderson2008sea}~\cite{Anderson2008sea} --- albeit with less 
technical details.
We also treat the results of \citetitle{BSIMMFindings}~\cite{BSIMMFindings}.
