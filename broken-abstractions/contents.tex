\mode*

% Since this a solution template for a generic talk, very little can
% be said about how it should be structured. However, the talk length
% of between 15min and 45min and the theme suggest that you stick to
% the following rules:  

% - Exactly two or three sections (other than the summary).
% - At *most* three subsections per section.
% - Talk about 30s to 2min per frame. So there should be between about
%   15 and 30 frames, all told.


\section{Broken Abstractions}

\subsection{File System Paths}

\begin{frame}[fragile]
  \inputminted{python}{jail.py}
\end{frame}

\begin{frame}[fragile]
  \begin{example}[./jail.py ../../etc/passwd]
    \begin{pycode}
import jail
jail.main(["jailopen", "../../etc/passwd"])
    \end{pycode}
  \end{example}
\end{frame}

\begin{frame}
  \pyc[variable]{import os}
  \begin{alertblock}{The Problem: Abstraction of paths}
    \begin{itemize}
      \item We had \pyb[variable]{JAIL_PATH = os.environ["HOME"]}.
      \item We let \pyb[variable]{filename = "../../etc/passwd"}.
      \item Thus the file we open is \pyb[variable]{JAIL_PATH + "/" + filename} 
        which results in \pyc[variable]{print(JAIL_PATH + "/" + filename)}.
      \item Hence we actually read /etc/passwd.
    \end{itemize}
  \end{alertblock}
\end{frame}

\begin{frame}
  \begin{itemize}
    \item Fine, we ban the string \mintinline{python}{"../"}.

    \item Then what about \mintinline{python}{"..\%c0\%af.."}?

  \end{itemize}
\end{frame}

\subsection{Character Encoding}

\begin{frame}
  \begin{itemize}
    \item All character representations in the computer comes in the form of 
      different encodings, e.g.\ UTF-8 encoding.

    \item The decoders might be programmed differently, some takes into account 
      the errors in different encoders to compensate -- and this can be 
      exploited.

    \item Where the encoding and decoding is done can also be exploited.

  \end{itemize}
\end{frame}

\begin{frame}
  \begin{block}{UTF-8}
    \begin{itemize}
      \item A character encoding standard.
      \item Uses variable length code words: from one byte.
      \item First bit indicates if next byte is part of the same code word.
    \end{itemize}
  \end{block}

  \begin{table}
    \begin{tabular}{rrllll}
      \textbf{Bytes} & \textbf{Avail bits} & \textbf{Byte 1}
                                           & \textbf{Byte 2}
                                           & \textbf{Byte 3}
                                           & \textbf{Byte 4} \\
                                           \toprule
      1 & 7  & 0xxxxxxx &          &          & \\
      2 & 11 & 110xxxxx & 10xxxxxx &          & \\
      3 & 16 & 1110xxxx & 10xxxxxx & 10xxxxxx & \\
      4 & 21 & 11110xxx & 10xxxxxx & 10xxxxxx & 10xxxxxx \\
      \bottomrule
    \end{tabular}
  \end{table}
\end{frame}

% XXX add more details on UTF-8 coding

\subsection{Integer Overflows}

% XXX add more examples on integer overflows
\begin{frame}[fragile]
  \inputminted{C}{combine.c}
\end{frame}

\begin{frame}
  \begin{alertblock}{The Problem: Abstraction of integers}
    \begin{itemize}
      \item Let \mintinline{C}{len2} be very long, say \(2^{32} - 1\), i.e.\ 
        \mintinline{C}{len2 = 0xffffffff}.

      \item Now we have
        \begin{align*}
          \text{\mintinline{C}{len1}} + \text{\mintinline{C}{len2}} 
          + 1 \pmod{2^{32}}
          &= \text{\mintinline{C}{len1}} + 2^{32} - 1 + 1 \pmod{2^{32}} \\
          &= \text{\mintinline{C}{len1}} \pmod{2^{32}} \\
          &< \text{\mintinline{C}{sizeof(buf)}}.
        \end{align*}

      \item Thus we pass the test, although we shouldn't.
    \end{itemize}
  \end{alertblock}
\end{frame}

\begin{frame}
  \begin{remark}
    This is worse if we use \emph{signed} integers \dots
  \end{remark}
\end{frame}

% XXX add more details and other examples of composition
\subsection{Data and Code}

\begin{frame}[fragile]
  \begin{example}[echo.sh "-E test\textbackslash ning"]
    \inputminted{sh}{echo.sh}
    \begin{pycode}[echo.sh]
import subprocess
proc = subprocess.Popen(["./echo.sh", "-E test\\ning"], \
stdout=subprocess.PIPE)
print("\\begin{verbatim}" + proc.stdout.read().decode("utf-8") + \
"\\end{verbatim}")
    \end{pycode}
  \end{example}
\end{frame}

\begin{frame}[fragile]
  \begin{example}[echofix.sh "-E test\textbackslash ning"]
    \inputminted{sh}{echofix.sh}
    \begin{pycode}[echofix.sh]
import subprocess
proc = subprocess.Popen(["./echofix.sh", "-E test\\ning"], \
stdout=subprocess.PIPE)
print("\\begin{verbatim}" + proc.stdout.read().decode("utf-8") + \
"\\end{verbatim}")
    \end{pycode}
  \end{example}
\end{frame}

\begin{frame}
  \begin{itemize}
    \item The login(1) and rlogin(1) composition bug was found in Linux and AIX 
      systems which didn't check the syntax of the username.

    \item The syntax of login(1) is \mintinline{sh}{login [-p] [-h host] [[-f] 
        user]}.

    \item The syntax of rlogin(1) is \mintinline{sh}{rlogin [-l user] machine}.

    \item rlogin(1) connects to the machine and runs \mintinline{sh}{login user 
        machine}.

    \item However, the user could be chosen to be \enquote{-froot}.
  \end{itemize}
\end{frame}

% XXX add canonical representations
%\subsection{Canonical Representations}
%
%\begin{frame}
%\end{frame}

% XXX add better description of scripting vuln

\begin{frame}[fragile]
  \begin{minted}{sh}
    cat ${1} | mail ${2}
  \end{minted}
  \begin{itemize}
    \item What happens with the address
      \mintinline{sh}{"foo@bar.org | rm -Rf /"}?
  \end{itemize}
\end{frame}

% XXX add better description and examples of SQL injection

\begin{frame}[fragile]
  \begin{minted}[startinline]{php}
    $sql = "SELECT * FROM client WHERE name = '$name'"
  \end{minted}

  \pause

  \begin{itemize}
    \item Insert the name \mintinline[startinline]{php}{Eve' OR 1=1--}.
    \item This will get a totally different meaning.
  \end{itemize}

  \pause

  \begin{minted}[startinline]{sql}
    SELECT * FROM client WHERE name = 'Eve' OR 1=1--
  \end{minted}
\end{frame}

\begin{frame}
  \begin{figure}
    \centering
    \includegraphics[width=\textwidth]{BobbyTables.png}
    \caption{%
      XKCD's Exploits of a Mom.
      Image: \cite{BobbyTables}.
    }
  \end{figure}
\end{frame}

