\question[3]\label{q:software}
% tags: software:E:C
We have talked about how the users' mental models of how a program (and 
computer) works can endanger the users' security when the mental model and 
reality are not aligned.
This is true also for developers (we mentioned this when we talked about 
software security), give an example of how the developers' mental models are 
relevant for software security.

\begin{solution}
  Gollmann talked about broken abstractions.
  One example is characters: usually we abstract away the encoding and decoding 
  parts, we see them as characters and not bytes.
  So encodings like UTF-8 can cause problems since the same character can be 
  represented by several byte sequences.

  Another is the finite precision arithmetic that we work with in computers, 
  usually congruences modulo \(2^{32}\) or \(2^{64}\).
\end{solution}

