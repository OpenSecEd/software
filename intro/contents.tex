\mode*

% Since this a solution template for a generic talk, very little can
% be said about how it should be structured. However, the talk length
% of between 15min and 45min and the theme suggest that you stick to
% the following rules:  

% - Exactly two or three sections (other than the summary).
% - At *most* three subsections per section.
% - Talk about 30s to 2min per frame. So there should be between about
%   15 and 30 frames, all told.


\section{Introduction}

\subsection{Security and Reliability}

\begin{frame}
  \begin{remark}
    \begin{itemize}
      \item As long as our computer is offline, used only by ourselves, and we 
        don't add any accessories (e.g.\ USB devices~\cite{ieeespectrum2014usb}), 
        then we don't have any problems.

        \pause

      \item Problems start to occur when other users start using our software (in 
        some way), then input to our programs isn't necessarily what we expect.

    \end{itemize}
  \end{remark}
\end{frame}

\begin{frame}
  \begin{description}
    \item[Software reliability] This concerns software quality in the sense of 
      accidental failures, i.e.\ the assumption that input is benign.

      \pause{}

    \item[Software security] This concerns software quality in the sense of 
      intentional failures, i.e.\ the assumption that input is malign.
  \end{description}
\end{frame}

\begin{frame}
  \begin{question}
    \begin{itemize}
      \item Test-driven development? C'est la mode.
    \end{itemize}
  \end{question}

  \pause

  \begin{solution}[BSIMM\footfullcite{BSIMMFindings}]
    \begin{itemize}
      \item Do code review.
      \item Have a Software Security Group (SSG).
      \item Integrate SSG into the organization (have a satellite).
    \end{itemize}
  \end{solution}
\end{frame}

\subsection{Changes}

% XXX add better storyline to changes

\begin{frame}
  \begin{remark}[Changes \dots]
    \begin{itemize}
      \item There are systems which are designed to be secure, and actually are 
        secure, but then \dots

      \item Upgrades needed, or, not needed but wanted.

      \item This might come in the form of updating a component or utilizing the 
        system in an environment it wasn't designed for.

    \end{itemize}
  \end{remark}
\end{frame}


